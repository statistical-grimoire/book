\begin{table}[htb]
\centering
\resizebox{\textwidth}{!}{%
%\begin{tabular}{l|l|p{5cm}|p{4cm}}
\begin{tabular}{l|l|>{\raggedright\arraybackslash}m{5cm}|>{\raggedright\arraybackslash}m{4cm}}
\toprule
\textbf{Variable} & \textbf{Scale} & \textbf{Description} & \textbf{Examples} \\
\midrule
\hyperref[sec:qual_var]{Qualitative} (Categorical) & \cellcolor{gray!10}{\hyperref[sec:nominal]{Nominal}} & \cellcolor{gray!10}{Categories with no inherent order.} & \cellcolor{gray!10}{Handedness, Eye colour, Species} \\
 & \hyperref[sec:ordinal]{Ordinal} & Values with a meaningful order, but differences between values are indeterminate. & Letter grades, Likert-scale ratings, Percentile ranks \\
 \midrule
\hyperref[sec:quant_var]{Quantitative} (Numerical) & \cellcolor{gray!10}{\hyperref[sec:interval]{Interval}} & \cellcolor{gray!10}{Numeric values with equal spacing, but no true zero} & \cellcolor{gray!10}{Temperature in Celsius, Calendar Years} \\
 & \hyperref[sec:ratio]{Ratio} & Numeric values with equal spacing and a meaningful zero & Height, Weight, Reaction time \\
\bottomrule
\end{tabular}
}
\caption{Summary of variables and corresponding measurement scales}
\label{tab:var_sum}
\end{table}
