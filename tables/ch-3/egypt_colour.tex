\begin{table}[!h]
    \centering
    \renewcommand{\arraystretch}{1.3} % Adjust row height for better readability
    \begin{tabularx}{\linewidth}{>{\hsize=1.2\hsize}X >{\hsize=0.7\hsize}X >{\hsize=0.8\hsize}X >{\hsize=0.6\hsize}X >{\hsize=1.7\hsize}X}
        \toprule
        Period & Approx. Date & Colour Name & Hex Code & Symbolism \\
        \midrule
        \rowcolor{gray!10} Early Pre-Dynastic & 4000 BCE & Red Ochre & \#A23E0E & Used in early rock art, pottery, and symbolizing vitality and the desert. \\
        Early Dynastic & 3300 BCE & Deep Green & \#20603D & Associated with fertility, rebirth, and Osiris. Found in early faience. \\
        \rowcolor{gray!10} Middle Kingdom & 1850 BCE & Lapis Lazuli & \#0C2C84 & Symbolic of divine power, seen in jewelry and sculpture. \\
        Ptolemaic & 200 BCE & Gold & \#D4AF37 & Representing wealth, the sun god Ra, and elite tombs. \\
        \rowcolor{gray!10} Roman & 150 CE & Sand Beige & \#E3C9A8 & Reflecting Greco-Roman influences, desert tones, and temple architecture. \\
        \bottomrule
    \end{tabularx}
    \caption{Ancient Egypt Colour Palette}
    \label{tab:egypt_colour}
\end{table}