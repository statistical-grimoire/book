%A-----------------------------------

\newglossaryentry{aesthetics}{
name=aesthetics,
description={The modifiable visual elements of a \textit{ggplot2} graph. E.g., point shapes, fill colours, edge colours, etc.}
}

\newglossaryentry{argument}{
name=argument,
description={Modifiable parameters of a function that alters how it operates.}
}

\newglossaryentry{assignment operator}{
name=assignment operator,
description={A symbol (e.g., \R{<-}) that assigns a name to an object in R so it can be easily sourced by the user from the computer's memory. R contains three different assignment operators. R Documentation: \R{?assignOps}}
}


%B-----------------------------------
\newglossaryentry{bivariate data}{
name=bivariate data,
description={Data consisting of a two variables.}
}

\newglossaryentry{boolean}{
name=boolean,
description={A term used to denote \gls{logical} (true or false) statements and objects. Named after the English mathematician and logician George Boole.}
}


%C-----------------------------------
\newglossaryentry{character}{
name=character,
description={A type of storage mode in R for character strings.}
}


\newglossaryentry{colon operator}{
name=colon operator,
description={A symbol, \R{:}, used to create regular sequences of integers. R Documentation: \R{?colon}}
}


\newglossaryentry{command console}{
name=command console,
description={An interface used for communicating instructions to a computer and (usually) viewing outputs. On modern digital computers it typically takes the form of a software application but, in ancient times, was a physical console of buttons and dials that you ``commanded'' the computer from.}
}

\newacronym{CRAN}{CRAN}{Comprehensive R Archive Network}

\newglossaryentry{Comprehensive R Archive Network}{
name=Comprehensive R Archive Network,
description={A set of mirrored servers around the world that distribute R and its associated packages.}
}

%D-----------------------------------
\newglossaryentry{data frame}{
name=data frame,
description={A object class in R with rows and columns resembling a spreadsheet structure. R Documentation: \R{?data.frame}}
}

\newglossaryentry{delimiter}{
name=delimiter,
description={A character within a data file used to delimit (i.e., define the limits of) individual values.}
}

\newglossaryentry{directory}{
name= directory,
description={An address that \textit{directs} you to a file}
}

%E-----------------------------------
\newglossaryentry{environment}{
name=environment,
description={see ``integrated development environment''}
}

\newglossaryentry{error bar}{
name=error bar,
description={A graphical representation of the variation surrounding a measure of central tendency. Displayed as lines (also called ``whiskers'') extending above and below a plotted value. They typically represent statistics such as the standard error or confidence intervals; however, they can, in principle, illustrate any statistic that conveys variation in the data.}
}

%F-----------------------------------

\newglossaryentry{factor}{
name=factor,
description={A class of object in R that has a defined set of possible values called \glspl{level}. Factors are used to represent categorical data, control the order of categories, and influence how data is processed or displayed in visualizations and models. R Documentation: \R{?factor}}
}

\newglossaryentry{file extension}{
name=file extension,
description={An identifier appended to the end of a file name that dictates how a file should be read by an application. The extension is indicated by a period followed by one to three characters. E.g., \texttt{my\_script.R} or \texttt{cat.png}}
}

\newglossaryentry{function}{
name=function,
description={A line of code that takes inputs (objects and \glspl{argument}) and produces a corresponding output.}
}

\newglossaryentry{functional}{
name=functional,
description={A function that accepts another function as an input and produces a vector as output. E.g., \R{apply()}}
}


%I-----------------------------------
\newacronym{IDE}{IDE}{integrated development environment}

\newglossaryentry{integrated development environment}{
name=integrated development environment,
description={A software application that aims to give programmers a nice visual workspace and comprehensive feature set with which to do their programming.}
}

\newglossaryentry{infinity}{
name=infinity,
description={Trying to define this is way above my pay grade (which for this textbook is literally nothing). Just see the ``Math is Fun'' website: \\ \url{https://www.mathsisfun.com/numbers/infinity.html}}
}

\newglossaryentry{Inf}{
name=Inf,
description={R's representation of infinity. Can be in either the negative direction \R{-Inf} or the positive direction \R{Inf}. Occasionally this will be generated if a number is too large for a computer to handle. E.g., \R{.Machine\$double.xmax * 2}
R Documentation: \R{?Inf}}
}

%L-----------------------------------

\newglossaryentry{level}{
name=level,
description={A category belonging to a \gls{factor} class of object.}
}

\newglossaryentry{logical}{
name=logical,
description={A type of storage mode in R for logical (i.e., true and false) values (also referred to as \gls{boolean} values).}
}


\newglossaryentry{logical operator}{
name=logical operator,
description={A symbol used to refine logical statements. R Documentation: \R{?Logic}}
}


%M-----------------------------------
\newglossaryentry{mode}{
name=mode,
description={A classification (e.g., numeric, character, logical) of how an object is stored in R.}
}

\newglossaryentry{modulo operator}{
name=modulo operator,
description={A mathematical operator that returns the remainder of a \textit{dividend} and \textit{divisor}.}
}

\newglossaryentry{modulus}{
name=modulus,
description={The value returned using a modulo operation.}
}

\newglossaryentry{multivariate data}{
name=multivariate data,
description={Data consisting of a more than two variables.}
}


%N-----------------------------------
\newglossaryentry{negation operator}{
name=negation operator,
description={Symbolized using a exclamation mark (\R{!}), this is a type of \gls{logical operator} that indicates the negation of an object's values.  For example, \R{!x} is read as ``not x.''}
}

\newglossaryentry{non-syntactic name}{
name=non-syntactic name,
description={A object name enclosed by backticks. E.g. \R{`fav num` <- 666}.}
}

\newglossaryentry{null value}{
name=null value,
description={Represented as \R{NULL} in the R language, this is used to represent undefined objects. R Documentation: \R{?NULL}}
}


\newglossaryentry{numeric}{
name=numeric,
description={A type of storage mode in R for numbers.}
}


%P-----------------------------------
\newglossaryentry{package}{
name=package,
description={A collection of functions, associated documentation, and data compiled for users to install via a online repository.}
}

\newglossaryentry{pch}{
name=pch,
description={R's abbreviation for ``plotting character''. An integer or character value that specifies what symbol gets plotted as a point on a graph.  R Documentation: \R{?points}}
}

\newglossaryentry{position scale}{
name=position scale,
description={In \textit{ggplot2}, this refers to a type of scale that controls the location mapping of a plot's visual elements.}
}

\newglossaryentry{programming language}{
name=programming language,
description={A language humans use to communicate instructions to a computer.}
}


%R-----------------------------------
\newglossaryentry{relational operator}{
name=relational operator,
description={A symbol (e.g., \R{==}) that is used to determine whether a specific comparison between two values is true or false. R Documentation: \R{?Comparison}}
}

\newglossaryentry{reserved words}{
name=reserved words,
description={Words that have specific functions and meanings within the R language and cannot be used as syntactic names. R Documentation: \R{?Reserved}}
}

\newglossaryentry{RStudio}{
name=RStudio,
description={An \gls{integrated development environment} for R.}
}

%S-----------------------------------
\newglossaryentry{scatter plot}{
name=scatter plot,
description={A type of graph that is used to visualize the relationship between two paired variables. The observations of one variable are plotted on the x-axis, while the observations of the other variable are plotted on the y-axis. The intersection of the x-y pairs are plotted as points on a Cartesian plane (i.e., a  grid). For further details see \url{https://www.mathsisfun.com/data/scatter-xy-plots.html}}
}

\newglossaryentry{scientific notation}{
name=scientific notation,
description={A method of writing very large or small numbers in a compact way. E.g., $66613000$ can be written as $666.13 \times 10^5$ or \R{666.13e+5}}
}

\newglossaryentry{script}{
name=script,
description={A text document (e.g., .R or .txt) for storing computer code that can be run or modified by a user. Integrated development environments usually provide a separate window for typing and saving scripts.}
}

\newglossaryentry{significant figures}{
name=significant figures,
description={The number of meaningful digits in a numerical value. Significant figures include all nonzero digits, any zeros between significant digits, and trailing zeros in decimal representations. They are also commonly referred to as significant digits.}
}

\newglossaryentry{subdirectory}{
name=subdirectory,
description={A directory nested within another directory.}
}

\newglossaryentry{syntactic name}{
name=syntactic name,
description={A object name consisting of letters, numbers and the dot or underline characters and starts with a letter or the dot not followed by a number.  R Documentation: \R{?make.names}}
}


%T-----------------------------------
\newglossaryentry{tibble}{
name=tibble,
description={The tidyverse's modern reimagining of the data frame.}
}

\newglossaryentry{tidy data}{
name=tidy data,
description={A sacred formation of data, guided by three precepts, that form the bedrock of the \gls{tidyverse}\textbf{'s} magik. Also referred to as the ``long format'' data by unbelievers.}
}

\newglossaryentry{tidyverse}{
name=tidyverse,
description={A powerful set of R magick, with an underlying philosophy, that allows those devoted to it to weave, transform, and manipulate data with a dark mystical ease that some call unnatural.}
}

%U-----------------------------------
\newglossaryentry{univariate data}{
name=univariate data,
description={Data consisting of a single variable.}
}


%V-----------------------------------
\newglossaryentry{vector}{
name=vector,
description={In R, a (atomic) vector is an object with at least one value and a single \gls{mode}. R Documentation: \R{?vector}\\ In computer programming more generally, a vector is a one-dimensional array of values.}
}


%W-----------------------------------
\newglossaryentry{wide format}{
name=wide format,
description={A way of structuring data such that variables are spread across multiple columns.}
}

\newglossaryentry{working directory}{
name=working directory,
description={The default address on a computer where R saves and pulls files.}
}